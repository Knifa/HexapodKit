\chapter{Implementation}

%%%%%%%%%%%%%%%%%%%%%%%%%%%%%%%%%%%%%%%%%%%%%%%%%%%%%%%%%%%%%%%%%%%%%%%%%%%%%%%%%%%%%%%%%%%%%%%%%%%%

This chapter will document the implementation process of the robot's systems.

At the time of writing, the most recent version of ROS is \emph{Hydro}. A machine running \emph{Ubuntu 12.04 LTS}, the primary distribution for Hydro, was used for all development.

%%%%%%%%%%%%%%%%%%%%%%%%%%%%%%%%%%%%%%%%%%%%%%%%%%%%%%%%%%%%%%%%%%%%%%%%%%%%%%%%%%%%%%%%%%%%%%%%%%%%

\section{Architecture}

Through the use of packages, we split the robot system into a number of systems, grouped by one larger system. Specifically, 

%%%%%%%%%%%%%%%%%%%%%%%%%%%%%%%%%%%%%%%%%%%%%%%%%%%%%%%%%%%%%%%%%%%%%%%%%%%%%%%%%%%%%%%%%%%%%%%%%%%%

\section{Hardware Operation}

% FIXME: Maybe put this elsewhere?
From the outset, it was known that the depth sensor relied on the \emph{OpenNI2} library and already had a supporting package. However, no such packages existed for the servo controller.

\subsection{RGB-D Camera Driver}

A wrapper for the \emph{OpenNI2} library is provided by the \texttt{openni2\_camera} \cite{ros_wiki_openni2_camera} and \texttt{openni2\_launch} \cite{ros_wiki_openni2_launch} packages. The former provides a single nodelet which acquires and publishes the image data, whereas the latter provides a means for starting that nodelet.

This driver publishes image data from the camera on a number of different topics, with varying data types. Only two of these are particularly useful to us, however. The topic \texttt{/camera/rgb/image} provides the general optical feed from the camera, and the topic \texttt{/camera/depth/image} provides the depth feed from the camera, both of message type \texttt{Image}. Some example output from these topics is shown below.

Additionally, a \texttt{PointCloud2} is published on \texttt{/camera/depth/points}.

Understanding these package was rather troublesome, as both the wiki and GitHub pages documenting them were (and still are) completely blank. Instead, the documentation for a similar set of these packages for the first version of \emph{OpenNI}, named \texttt{openni\_launch}, was used \cite{ros_wiki_openni_launch}. Regardless, there was little difficulty in setting this package up.

\subsection{Servo Driver}

While there were already drivers for the RGB-D sensor, the case was not the same for the servo controller. For this, a custom node had to be developed.

%%%%%%%%%%%%%%%%%%%%%%%%%%%%%%%%%%%%%%%%%%%%%%%%%%%%%%%%%%%%%%%%%%%%%%%%%%%%%%%%%%%%%%%%%%%%%%%%%%%%

\section{Locomotion}

Strictly, only joints 1 and 2 are necessary for a tripod gait. Joint 3 can be locked, perhaps as a single solid piece of material, such that a right angle is formed between the foot and leg sections. This provides a stable stance for the robot to balance upon as it pushes itself forward between each cycle.

\subsection{Limb Controller}
\subsection{Limb Calibration Tool}
\subsubsection{Usage}

\subsection{Tripod Gait Walker}
\subsection{Joystick Controller}

%%%%%%%%%%%%%%%%%%%%%%%%%%%%%%%%%%%%%%%%%%%%%%%%%%%%%%%%%%%%%%%%%%%%%%%%%%%%%%%%%%%%%%%%%%%%%%%%%%%%

\section{Sensing}

openni2 originally, ccny\_rgbd \cite{ccny_rgbd} provides some clean up features.

\subsection{Visual Odometry}

ccny\_rgbd \cite{ccny_rgbd} was used.

\subsection{Environment Mapping}

ocotomap

\subsubsection{Alternatives}

SLAM, but expects laser scan.

%%%%%%%%%%%%%%%%%%%%%%%%%%%%%%%%%%%%%%%%%%%%%%%%%%%%%%%%%%%%%%%%%%%%%%%%%%%%%%%%%%%%%%%%%%%%%%%%%%%%

\section{Navigation}

Built in stack.

\subsection{Path Planning}
