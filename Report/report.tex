\documentclass{l4proj}

\usepackage{alltt}
\usepackage{microtype}
\usepackage{textcomp}
\usepackage[small,bf]{caption}
\usepackage{subcaption}
\usepackage{multirow}
\usepackage{enumitem}
\usepackage{booktabs}

\usepackage[usenames, dvipsnames]{color}
\usepackage{listings}
\lstset{
	basicstyle=\ttfamily
}

\usepackage{graphicx}
\graphicspath{{./images/}}

\usepackage[]{hyperref}
\hypersetup{
    colorlinks = true,
    citecolor = black,
    linkcolor = black,
    urlcolor = black,
}

%%%%%%%%%%%%%%%%%%%%%%%%%%%%%%%%%%%%%%%%%%%%%%%%%%%%%%%%%%%%%%%%%%%%%%%%%%%%%%%%%%%%%%%%%%%%%%%%%%%%

\begin{document}

\title{Visual Control of a Hexapod Robot}
\author{Daniel Callander}
\date{March 28, 2014}
\maketitle

%%%%%%%%%%%%%%%%%%%%%%%%%%%%%%%%%%%%%%%%%%%%%%%%%%%%%%%%%%%%%%%%%%%%%%%%%%%%%%%%%%%%%%%%%%%%%%%%%%%%

\begin{abstract}
Robot control systems require a large underlying infrastructure of software to facilitate any complex behaviour. Typical systems require a means for communicating with hardware, locomotion control for movement, vision systems for guidance and environment mapping, autonomous navigation, means of user interaction, and so on. Each of these components requires a deep understanding of the subject matter and an immense number of person hours to design, implement, and test.

We implement a robot control system for a hexapod-type robot using the Robot Operating System (ROS) as a framework, an open-source framework designed specifically for building robot control systems. Through this implementation, we show that it is possible to endow a robot with these complex behaviours---specifically visual odometry, environment mapping and autonomous navigational functionalities---without the need to invest a large number of person hours. The key advantage of ROS is its vast amount of standard and community-provided packages which implement the necessary functionality for these complex behaviours. We make great use of these packages, offloading the complexities of development to those who are more qualified. 

Furthermore, we show that it is possible to implement these behaviours on a robot built using only consumer components rather than expensive research hardware. An RGB-D camera is used as the primary sensor for understanding the environment, which is used to provide mapping and positional information throughout the control system.
\end{abstract}

\renewcommand{\abstractname}{Acknowledgements}
\begin{abstract}
I would like to thank Dr. Paul Siebert for providing me with his invaluable guidance and expertise throughout the project. It would have been very tough going without his help! 

\noindent
I would also like to thank my fantastic friends and family for their support during the final stages of the project, especially for their proof-reading services for which I am extremely grateful.

\noindent
Finally, it would be rude of me not to thank the robot in some form for remaining (mostly) intact throughout the project. Even though the legs fell off a few times, everything remained working right to the very end.

--- Daniel Callander, 2014
\end{abstract}

\educationalconsent
\tableofcontents

\chapter{Introduction}
\label{chap:introduction}
\pagenumbering{arabic}

%%%%%%%%%%%%%%%%%%%%%%%%%%%%%%%%%%%%%%%%%%%%%%%%%%%%%%%%%%%%%%%%%%%%%%%%%%%%%%%%%%%%%%%%%%%%%%%%%%%%

The objective of this project was to implement a control system for an existing hexapod-type robot (shown in \autoref{fig:hexapod}) through the use of the Robot Operating System (ROS) \cite{ros_site}. By attempting to implement a number of complex behaviours such as environment mapping and autonomous navigation, we aimed to evaluate the usefulness of ROS as a framework for rapidly developing these control systems. Specifically, we looked to explore a range of standard and community-provided packages available for use through ROS which provide state-of-the-art functionality, including those that were necessary for this control system.

\begin{figure}[!h]
    \centering
    \includegraphics[width=12cm]{hexapod_body1}
    \caption{This hexapod-type robot which is used as the target hardware platform for this project. The robot has six limbs with three actuators on each, for a total of eighteen actuators. A RGB-D camera sits atop as the primary sensor, providing the robot with a 3D view of its immediate environment. A controller sits within the base of the robot, providing the necessary signals to operate the actuators. A bundle of cables protrudes from the back connecting the hardware to a nearby computer for operation, as well as a power source.}
    \label{fig:hexapod}
\end{figure}

This report will detail how these objectives were achieved. Through the use of ROS and its packages, it was possible to implement a control system that provides a number of these complex behaviours with relative brevity and simplicity. ROS endows the resulting robot control system with a number of impressive capabilities:

\begin{description}[labelindent=\parindent]
	\item[Hardware Operations] \hfill \\
	The system is capable with interacting with the RGB-D camera and servo controller on-board the robot. Commands can be sent through the system to the controller, allowing for the movement of any particular actuator. Image data from the sensor is received, processed, and then broadcast throughout the system, such that any subsystem needing the imagery can access it.

	\item[Locomotion] \hfill \\
	The system allows the robot to walk with a tripod-type walk gait in an abstracted manner. This abstraction allows any control subsystems to request linear and angular movements, without regard as to the specific actuator commands. The system also provides a means for calibrating the actuators, allowing for movements to be made precisely. Additionally, the robot can be manually controlled via a standard game controller.

	\item[Sensing] \hfill \\
	The system is capable of building a 3D map of the environment around the robot, updating in real-time as the robot moves around. This map is stored in an efficient manner, allowing the system to quickly determine if any obstacles are nearby. Additionally, the system is capable of localising the position of the robot relative its environment.

	\item[Navigation] \hfill \\
	The system is capable of fully autonomous navigation. Given a target position, the system will determine the most efficient path through the known environment and then supply the appropriate commands to move the robot to that position. Should the robot encounter any new obstacles, the path will be recalculated in real-time to avoid that obstacle.

\end{description}

Without access to the packages available for use through ROS, each of these capabilities would have taken a great number of person hours to develop, test and integrate. 

%%%%%%%%%%%%%%%%%%%%%%%%%%%%%%%%%%%%%%%%%%%%%%%%%%%%%%%%%%%%%%%%%%%%%%%%%%%%%%%%%%%%%%%%%%%%%%%%%%%%

\section{Motivation}

The primary motivation was to show that it is possible to implement these complex behaviours, previously restricted to research projects with large budgets, on robots utilising cheaper consumer components. Laser range-finders, in particular, are typically used for environment scanning and positional tracking. These scanners provide extremely accurate results but can cost multiple thousands of pounds. However, as RGB-D sensors become increasingly available, it is possible to achieve similar results at a fraction of the cost with reduced accuracy.

As a secondary motivation, we wanted to show that it was possible to implement these behaviours without having to invest an immense number of person hours. Through the use of readily available libraries and frameworks, it is possible for a very small team to implement these behaviours without having to fully understand the underlying intricacies of any one system. Robot control systems are, simply, software development projects. By applying standard software development concepts, such as code reuse and separations of concerns, it allows us to make use of any existing works without having to ``re-invent the wheel'' each time.

Additionally, these behaviours, such as being able to move around and understand an environment, act as a general foundation for further developments. The ramifications of being able to rapidly develop these robot system are such that new developments and innovations can be made more quickly, with less overall cost in terms of both budget and time.

\section{Overview}

To give further context to the project, we provide the following overview of the control system. The control system is divided into a number of subsystems, each dealing with a particular task necessary for achieving the behaviours described in the opening text. Each subsystem is additionally divided into more discrete nodes, as discussed in the later sections. These subsystems are referred to throughout the report and are described briefly below.

\begin{description}[labelindent=\parindent]
	\item[Hardware Operations] \hfill \\
	This subsystem provides the necessary drivers to communicate with the robot hardware, converting any device-specific data into more generic formats that can be used elsewhere in the system.

	\item[Locomotion] \hfill \\
	This subsystem provides the robot with a means of moving itself around the environment in an abstracted manner, hiding the complexities of any actuator control.

	\item[Sensing] \hfill \\
	This subsystem interprets sensory data in a meaningful way, building up a view of the environment around the robot.

	\item[Navigation] \hfill \\
	This subsystem combines the sensing and locomotion subsystems to achieve autonomous navigation through the environment surrounding the robot.
\end{description}

ROS is used as a framework for implementing this control system, providing a means of communication between each component. Where possible, built-in and community-provided ROS packages are used to provide the implementation, rather than relying on any self-sourced code. A high-level overview of the system is shown in \autoref{fig:overview}, showing the connectivity between each subsystem and the underlying hardware it relies on. Upon starting the system, each subsystem begins communicating such that the complex behaviours can be performed by relying on data from one another.

\begin{figure}[!h]
	\centering
	\includegraphics[width=10cm]{overview.png}
	\caption{This block diagram shows a high-level view of the robot control system. Arrows represent the communication channels between the different subsystems and hardware. Notably, communication between the hardware operation, sensing, and navigation subsystems is unidirectional only, as there is no sensor control available.}
	\label{fig:overview}
\end{figure}

As with any good software development project, this separation of concerns is essential for building a manageable system. The methods with which each subsystem communicate is generic, such that one can be replaced in its entirety with some other implementation, as long as it uses a similar interface. This also means that the subsystems can be used in other projects, should they be relevant.

%%%%%%%%%%%%%%%%%%%%%%%%%%%%%%%%%%%%%%%%%%%%%%%%%%%%%%%%%%%%%%%%%%%%%%%%%%%%%%%%%%%%%%%%%%%%%%%%%%%%

\section{Outline}

The remainder of this report is divided into the chapters described below.

\begin{description}[labelindent=\parindent]
	\item[Chapter~\ref{chap:background}: Background] \hfill \\
	This chapter elaborates on the context of the project, specifically the technical intricacies of ROS and the robot hardware. A number of existing robot systems will also be presented for comparison.

	\item[Chapter~\ref{chap:approach}: Approach] \hfill \\
	This chapter details the development approach used throughout the project, including the requirements for the robot control system.

	\item[Chapter~\ref{chap:implementation}: Implementation] \hfill \\
	This chapter explains the resulting implementation of the robot control system, elaborating on any design decisions or caveats.

	\item[Chapter~\ref{chap:evaluation}: Evaluation] \hfill \\
	This chapter presents the results on an evaluation of the performance of the robot control system, as well as detailing the experimental setup.

	\item[Chapter~\ref{chap:usage}: Usage] \hfill \\
	This chapter shows how the implemented robot control system is used. Specifically, how a user can interact with the control system, allowing navigational commands to be given.

	\item[Chapter~\ref{chap:conclusion}: Conclusion] \hfill \\
	This chapter summarises the report, providing a number of ideas for future development of the project.
\end{description} 
\chapter{Background}
\label{chap:background}

For any advanced robot to operate correctly, a vast infrastructure of software is generally required. Developing this directly without the use of any frameworks or libraries would require a large amount of person hours, in terms of both design, implementation and integration time. Many different subsystems are required, each of which would require a deep technical understanding of the underlying intricacies. Typical systems require a means for communicating with hardware, locomotion control for movement, vision systems for guidance and environment mapping, autonomous navigation, means of user interaction, and so on. The complexity of each of these is such that they are research fields in their own right. 

However, through the use of ROS it is possible to gain access to these state-of-the-art functionalities free of charge, both in terms of cost and in person hours. The open-source nature of ROS is such that any innovations made by other developers can be easily shared to with the rest of the community. This offloads the complex understanding to smaller development teams which have expertise in that particular area.

The following sections in this chapter elaborate on the features of the ROS and our robot hardware, explaining why they are relevant to this project. This will provide a basis of understanding necessary for the implementation chapter and beyond.

%%%%%%%%%%%%%%%%%%%%%%%%%%%%%%%%%%%%%%%%%%%%%%%%%%%%%%%%%%%%%%%%%%%%%%%%%%%%%%%%%%%%%%%%%%%%%%%%%%%%

\section{Robot Operating System}

The \emph{Robot Operating System} (ROS) is an open-source framework for creating robot control systems, developed by \emph{Willow Garage} \cite{ros_paper}. ROS can be primarily thought of as a message-passing framework. A number of discrete processes, perhaps some even on a remote machine, communicate through a single broker service (the ROS "master") using an XMLRPC-based API \cite{ros_paper}. Each process performs a small part of a larger task, sharing data with other nodes through common data types.

ROS functions as a distributed system by default. Specifically, the processes can run on a machine different from that of the broker. The processes can then connect to the broker over a network, receiving data seamlessly from any other processes in the system. In cases where nodes are running on the same machine as the broker, they are simply connecting to the local machine. This can be used to offload complex computations onto another machine as necessary. To give an example, an embedded system on a robot running ROS can be collecting sensor data and issues hardware commands. This embedded system can then be communicating to a more powerful machine, acting as a base station. This base station can perform the more complex operations such as sensor data interpretation and path planning, relaying the commands back to the embedded system to be carried out.

\subsection{Supported Platforms}
At the time of writing, the latest version of ROS is \emph{Hydro Medusa} which primarily targets the \emph{Ubuntu} Linux distrubution, specifically versions 12.04 (LTS) to 13.04 \cite{ros_wiki_installation, ros_wiki_installation_ubuntu}. An Ubuntu repository containing binaries for ROS and its standard library of packages is available allowing for a trivial installation process, as well as allowing any updates to be delivered and installed with simplicity.

Experimental support is also available for other Linux distributions and systems, such as Ubuntu ARM, OS X, Debian, Arch Linux, Windows, and a number of embedded platforms \cite{ros_wiki_installation}. As the source code for ROS is freely available, it is possible to compile it on any system as long as the relevant supporting libraries are available.

\subsection{Nodes}
In ROS terminology, a processes is referred to as a \emph{node}. A node is simply a process that is connected to the ROS broker, performing some sort computation \cite{ros_paper}.

Nodes need not be tied to any particular programming language \cite{ros_paper}. Nodes are developed using a particular \emph{client library}, of which many are available. These are libraries developed for a particular language or ecosystem which provide an abstraction layer for interfacing with ROS. The two most used client libraries are \emph{roscpp} and \emph{rospy} which target C++ and Python respectively \cite{ros_wiki_clientlibraries}. There are also a number of experimental client libraries available for other systems such as Java, Android, C\#, and Arduino, among others \cite{ros_wiki_clientlibraries}.

\begin{figure}[!h]
    \centering
    \includegraphics[width=16cm]{nodes.png}
    \caption{A simplified diagram showing an example of node relations, modelled after the implemented robot system. White boxes represent particular nodes, while the grey boxes surrounding them represent a namespace in which they exist. Arrows between nodes indicate a communication flow.}
    \label{fig:nodes}
\end{figure}

As intraprocess communication can be somewhat costly in terms of performance, ROS provides a means for running multiple nodes in the same process via \emph{nodelets} \cite{ros_wiki_nodelet}. This allows messages to be passed in a ``zero-copy'' manner, as pointers to memory locations for any messages can be shared directly. Nodes must be modified to support this behaviour, however, as they must export some class that can be dynamically loaded by a \emph{nodelet manager}. This feature is used extensively when real-time performance is required such as with computer vision related tasks, for example.

\subsection{Topics \& Services}
Node communication is done through \emph{topics} in a one-to-many fashion \cite{ros_paper}. Broadcasting is done asynchronously, such that a broadcasting node can continue operating without regard as to which nodes receive the message. Any other node can then subscribe to that particular topic, receiving messages through a callback for processing. Each topic has a given resource name which is used to uniquely identify that topic.

A topic is defined with a given \emph{message} type through \emph{msg} files. Messages are simple data structures consisting of a number of typed data fields. Each data field has a unique name and can be of a number of built-in primitive types, as well as other arbitrarily nested data structures such as arrays and even other messages. A simple example of this would be the \texttt{Point} message from the \texttt{geometry\_msgs} package. This message consists of three \texttt{float64} fields named \texttt{x}, \texttt{y}, and \texttt{z} \cite{ros_api_point_msg}.

ROS also provides a \emph{request-response} style method of communication through \emph{services} \cite{ros_wiki_services}. In this case, a \emph{service} type is defined in a format similar to topic messages through \emph{srv} files. However, the difference is that two messages are defined in one file---one for the request and the other for the response. This communication method is synchronous in that a requesting node will block until a response is given. 

\subsection{Resource Names}

Each resource in the system has a unique name and can be placed into a particular namespace. This gives each resource a unique \emph{graph resource name} which can be used to identify this resource elsewhere throughout the system. As the name suggests, the layout of a ROS system can be thought of as a graph or tree. Some examples of \emph{graph resource names} are shown in \autoref{tab:graph_resource_names}.

\begin{table}[!h]
    \centering
    \begin{tabular}{  l l  }
        \toprule
        \textbf{Path} & \textbf{Description} \\
        \midrule
        \texttt{/} & the global namespace \\
        \texttt{/hexapod/} & the \texttt{hexapod} namespace \\
        \texttt{/hexapod/servo\_controller} & a \texttt{servo\_controller} node \\
        \bottomrule
    \end{tabular}
    \caption{Some examples of graph resource names.}
    \label{tab:graph_resource_names}
\end{table}

Nodes, topics, and services must all have unique \emph{graph resource names} to identify them within the system. Using namespaces, it is possible to run multiple instances of a particular node such that they do not conflict with each other. Any topics that they would use can also be remapped to point to one that is more appropriate for its usage.

\subsection{Packages}
A collection of nodes providing a particular set of functionalities---e.g., path planning for autonomous navigation---can be grouped and distributed as \emph{packages} \cite{ros_paper}. This can be used to divide nodes in a control system into logical subsystems. By developing nodes and packages in a generic way, they can also act as a means for providing drop-in functionality.

Users are encouraged to share and distribute any developments they make by hosting repositories of their code, preferably on \emph{GitHub} \cite{ros_wiki_getinvolved}. Contributors can then request that their repository be listed on a package index on the ROS website \cite{ros_wiki_getinvolved}. This aspect is a particular advantage of ROS as there is a wide array of packages available for usage. Additionally, as ROS provides a number of common data types used in robotics, packages from different vendors can interact with one another with relative ease.

\subsection{Standard Packages \& Utilities}
The standard ROS distribution contains a number of standard packages and utilities. These provide a number of essential features that streamline the entire development process. A number of these are particularly useful for use with our robot control system.

\subsubsection{roslaunch: Programatically Start Nodes}
While nodes can launched manually, ROS provides a means for large sets of nodes programmatically via \emph{roslaunch} \cite{ros_paper, ros_wiki_roslaunch}. This tool allows developers to specify a set of nodes to be ran, along with a number of parameters, in XML configuration files (``launch files'' in ROS terminology). \emph{roslaunch} provides a number of useful features:

A key feature is the ability to specify which namespace a particular set of nodes is in as mentioned previously. This allows multiple nodes of the same type to be launched without conflict. Any topic used by a node can also be remapped to one more appropriate. This means that nodes can be built generically, assuming that their topics will be remapped to fulfil some specific purpose. An example of this is the \texttt{depthimage\_to\_laserscan} node, which converts a depth image into a format usable by nodes expecting laser scan data. This node expects the depth image to be transmitted on an \texttt{image} topic assuming that this will be remapped to the actual depth image.

Parameters can also be specified in a standard format which the nodes can then read from on launch. This allows any node to be reconfigured as required as long as it provides a means of doing so, making any implemented node much more flexible and general purpose. Without this, nodes would have to result to using some other method of configuration such as domain-specific files or even require their source code to be changed---neither of which would function particularly well in large systems.

Launch files can also include other launch files. Generally, a package will provide a launch configuration that starts all the necessary nodes to provide dealing with this package. This can be used to create a ``master'' configuration that launches a number of subsystems to implement some overall control system.

\subsubsection{rviz: Visualising Sensor Data}

In addition to providing autonomous behaviour, it is often necessary that a control system must be able to provide some useful visualisations to the user operating the system. This can be much more helpful for gaining an understanding of what data the system is receiving versus trawling through text logs alone. Rather than having to develop new visualisation software for every robot system, ROS provides a generic means for achieving this via \emph{rviz} \cite{ros_wiki_rviz}.

\begin{figure}[!h]
    \centering
    \includegraphics[width=15cm]{rviz1.jpg}
    \caption{An example configuration of RViz, specifically in the configuration used for the robot control system. The two image feeds from the camera can be seen on the left hand side, along with the configuration window for each visualisation. The main window shows the space in which the robot is operating, where a number of visualisations show the positioning and mapping facilities of the system. The green line, in particular, represents the current navigational path that the robot will follow.}
    \label{fig:rviz}
\end{figure}

\emph{rviz} is a highly configurable program for displaying visualising data from topics in a control system. It can be considered the hub of interactivity for the robot control system, allowing a user to control and understand the underlying control system. Visualisations are made possible through a plugin system, a number of which are distributed with the package. These plugins provide visualisations for a number of common messages \cite{ros_wiki_rviz_datatypes}, examples of which are shown in \autoref{fig:rviz}. A major advantage of this plugin concept is that new visualisations can be developed an integrated with \emph{rviz} easily without having to modify the core of the program itself. This may be necessary if a control system implements some new type of message, for example. In particular, many community-provided packages implement new visualisations for their custom messages.

Additionally, it is possible to interact with the control system through \emph{rviz} through \emph{interactive markers} \cite{ros_wiki_rviz_intmark}. A user is able to move or rotate these markers on-screen, depending on the configuration, which can be used to control a corresponding robot limb, for example. The same markers can also be used to set navigational waypoints which a robot can then follow. Internally, marker data is transmitted and received through the usual topic system.

\subsubsection{tf: Transforming Between Various Coordinate Frames}
A common problem in robotics is the need to transform between different coordinate frames. For example, a number of laser rangefinders may be placed in varying positions around the the base of a robot. The incoming range data from a rangefinders will be relative to its origin. Should it be necessary to find the distance from the base of the robot to a nearby wall in sight of the rangefinder, for example, some arithmetic must be performed to calculate the distance, based on the physical offsets between the base and the rangefinder. Rather than performing these calculations manually each time, ROS provides a standard solution for achieving this through the \emph{tf} package \cite{ros_wiki_tf}. An annotated example of these concepts is shown in \autoref{fig:tf}.

\begin{figure}[!h]
    \centering
    \includegraphics[width=17cm]{tf2.png}
    \caption{A diagram showing the necessity for some sort of transformation system, modelled after the robot. In the first picture, the two transformation links for this robot are shown. In the second picture, the distances between the origins of the two transformation links are shown. Specifically, it can be said that \texttt{camera\_link} is positioned at an offset of $(0.15, 0.30, 0)$ meters relative to \texttt{base\_link}. The inverse of this is also true in that \texttt{base\_link} is positioned at an offset of $(-0.15, -0.30, 0)$ meters relative to \texttt{camera\_link}. In the third picture, points detected on a wall at a distance of $(0.2, 0, 0)$ meters relative to the camera are shown. By applying the transformation offsets between \texttt{camera\_link} and \texttt{base\_link}, it is possible to calculate the distance of the points relative to base of the robot. Through simple arithmetic via the transform, it can be shown that the points are at a position of $(0.35, 0.3, 0)$ meters relative to the base of the robot.}
    \label{fig:tf}
\end{figure}

The relations between different coordinate frames and their transformations are represented in a tree-like hierarchy where each node corresponds to a particular coordinate frame. A fixed frame is used as the root of this node representing a coordinate frame from which the rest can relate to. This fixed frame can represent the environment around a robot, for example. By traversing through this tree, it is possible to calculate positions in one coordinate frame relative to another. An example transform tree is shown in \autoref{fig:tf_tree}.

\begin{figure}[!h]
    \centering
    \includegraphics[width=10cm]{tf.png}
    \caption{A simplified diagram of a transform tree modelled after the implemented robot control system. Grey boxes represent a particular transform, while arrows between boxes indicate a transform relation. Each transform provides its position and rotation in space relative to the previous transform. For example, it is possible to get the position of the RGB-D camera relative to the environment by requesting a transform between \texttt{camera\_link} and \texttt{map}. The \emph{tf} system will compute the resulting position based on the supplied transforms, working backwards through the hierarchy as necessary.}
    \label{fig:tf_tree}
\end{figure}

The \emph{tf} package has a variety of useful features. Transform positions are cached throughout run-time such that it is possible to request positions at times in the past. Additionally, a number of nodes are available in the \emph{tf} package that can publish static transforms between coordinate frames. In the example shown in \autoref{fig:tf}, one would use these nodes to publish the offsets between base of the robot and the camera as they remain in the same relative position throughout run-time.

%%%%%%%%%%%%%%%%%%%%%%%%%%%%%%%%%%%%%%%%%%%%%%%%%%%%%%%%%%%%%%%%%%%%%%%%%%%%%%%%%%%%%%%%%%%%%%%%%%%%

\section{Hardware}

The robotic hardware used in this project is relatively straightforward, consisting of a number of off-the-shelf components. The robot is a hexapod-type in that it has six limbs. Each limb has three joints which can be rotated as shown in \autoref{fig:hexapod_dof}, giving three degrees-of-freedom per limb. 

\begin{figure}[!h]
    \centering
    \includegraphics[width=12cm]{hexapod_joints}
    \caption{This annotated photograph shows the three degrees-of-freedom available to each limb on the robot. Joint 1 provides horizontal rotation to the leg, which can be used to push the robot forward. Joints 2 and 3 provide vertical rotation which can be used to adjust the robot's height, as well as being used for locomotion in general.}
    \label{fig:hexapod_dof}
\end{figure}

In its current state, the robot has no wireless capability and, thusly, operates in a tethered manner. A bundle of cables protruding from the rear end of the unit connects the on-board hardware to a nearby power source and computer.

\subsection{Servos \& Servo Controller}
Each joint piece is attached to the shaft of a servomechanism (servo) to facilitate rotation. Each servo is controlled by a \emph{pulse-width modulated} (PWM) signal supplied by the controller such that the angle of the servo can be controlled precisely. Servos operate using a feedback-loop system such that the current of an internal motor is controlled, rotating the shaft to the desired position. Rotational range and speed of servos vary depending on model, but in this case the servos allow for a total rotational range of 180\textdegree.

The servos are controlled by an off-the-shelf servo controller board. This board, in particular, expects text-based control commands over serial, either via direct TTL or emulated over an on-board USB to TTL converter \cite{torobot_manual}. A number of commands are supported, allowing control of the position of both single and groups of servos. Additionally, a set of movements can be programmed and issued with a single command \cite{torobot_manual}. Power is also distrusted to the servos via this board.

\subsubsection{Protocol}

Commands are issued using an ASCII-based protocol via serial communication. While the servo controller supports a number of commands, only the command which rotates an individual servo is necessary. This allows for more flexibility in terms of when the rotation begins, as all servos mentioned in a group command will begin rotation at the same time.

\begin{figure}[!h]
    \centering
    \texttt{\#\(n\)\#P\(a\)T\(t\)}
\end{figure}

A rotation command is issued by sending a string in the format shown above, specifying the particular servo number ($1 \leq n \leq 32$), the target position ($1500 \leq a \leq 2500$), and the time over which the rotation should occur ($100 \leq t \leq 9999$) in milliseconds. The command is ended by a carriage return followed by a new line character. A working example is shown below, where we rotate the servo at index 1 to 90\textdegree{} over 250ms.

\begin{figure}[!h]
    \centering
    \texttt{\#1\#P2000T250}
\end{figure}

It should be noted that the parameter for the target position is actually the pulse-width duration (in milliseconds) that is sent to the servo rather than an actual angle. Duration ranges can very per manufacturer however there is a common standard such that 1.5\textmu s corresponds to 0\textdegree{} and 2.5\textmu s correspond to 180\textdegree. These are the minimum and maximum possible values for this controller. 

\subsection{RGB-D Camera}
The primary sensor in the system is an \emph{ASUS Xtion Pro Live} RGB-D camera, which is very similar to the \emph{Microsoft Kinect}. The key functionality of this camera is that it provides a depth data feed, giving a range of values indicating the distances to the objects in front of it. The images shown in \autoref{fig:rgbd_images1} show an example of the output provided by this camera. A combination of an infra-red grid emitter and infra-red sensor is used to calculate the distances of any nearby objects, however we need not concern ourselves with the particular intricacies of its operation.

\begin{figure}[!h]
    \centering
    \includegraphics[width=8cm]{rgbd_rgb2.jpg}
    \includegraphics[width=8cm]{rgbd_depth2.jpg}
    \caption{Example imagery from the RGB-D camera mounted on top of the robot as given by the \texttt{openni2\_camera} package. The image on the left shows the RGB output from the camera. The image on the right shows the depth output interpreted as a monochrome image.}
    \label{fig:rgbd_images1}
\end{figure}

It should be noted that the depth detection range is not unlimited. The specification for the device states that the range of detection is from $0.8$m to $3.5$m \cite{xtion_spec}. Objects that are too close will cause significant errors in the resulting depth imagery, while objects that are too far away will simply not be detected. Additionally, the device has problems with reflective surfaces, again causing significant errors in the resulting depth imagery. These issues do not effect the colour feed.

The advantages this particular sensor provides over the \emph{Kinect} is mostly physical, specifically weight and footprint. The \emph{Kinect} has a motorized base which allows the device to be tilted upwards and downwards through software, in comparison to the \emph{Xtion} which has a simple hinge that must be rotated by hand. This feature is unnecessary in this use case and adds a significant amount of weight. Additionally, the \emph{Kinect} is intended to be a consumer device and, thus, has a much striking product design. However, this striking design comes at the cost of making the device much larger in general. As space on the robot's base is at a premium, this makes the \emph{Xtion} much more favourable.

Should one wish to develop for the \emph{Xtion} in general, they must use the \emph{OpenNI} framework which is a ``standard framework for 3D sensing'' \cite{openni_site}. In particular, the \emph{Xtion} requires the use of \emph{OpenNI2} which is the second iteration of this framework. The framework encapsulates much of the complex hardware interaction such that it is possible to get the output imagery with relative ease. 

%%%%%%%%%%%%%%%%%%%%%%%%%%%%%%%%%%%%%%%%%%%%%%%%%%%%%%%%%%%%%%%%%%%%%%%%%%%%%%%%%%%%%%%%%%%%%%%%%%%%

\section{Existing Examples}
\chapter{Approach}
\label{chap:approach}

To evaluate the usefulness of ROS, we set out to build a ROS-based robot control system, targeting the hexapod as its primary hardware platform. A number of increasingly complex requirements were necessary for this system to function correctly.

Throughout, an attitude of code re-use was taken, relying on any existing nodes and packages where possible. It was assumed that these requirements would be otherwise unachievable in this project's time frame without the help of this vast package repository. Packages would be found by simply searching on the web, particularly on the ROS wiki which is used as a centralised resource for all ROS documentation.

%%%%%%%%%%%%%%%%%%%%%%%%%%%%%%%%%%%%%%%%%%%%%%%%%%%%%%%%%%%%%%%%%%%%%%%%%%%%%%%%%%%%%%%%%%%%%%%%%%%%

\section{Requirements}

The requirements for the control system can be divided into four key categories as explained in the following sections.

\subsection{Hardware Operation}

To achieve any real-world interaction, the system had to be capable of interfacing with our target hardware platform. Drivers in some form were required to allow communication to both the servo controller and the RGB-D camera. Ideally, these should transmit and receive on topics using common data types such that they can interact with any generic libraries.

\subsection{Locomotion}

As this robot relies on walking motions for movement, which involves a complex sequence of servo rotations, some abstraction of this process was required. Specifically, it was necessary to be able to control both the linear and angular velocities of the robot without knowledge of these underlying servo movements.

Some calibration process was also necessary. Due to the manner in which the servos connect to their respective joint sections, it is extremely difficult to align them such that they are perfectly straight. A method was required to specify offsets to any positional commands sent to the servo controller on a per-servo basis. Furthermore, to make this process easier some tool to adjust these offsets would be required.

Additionally, a method of manually controlling the robot was needed. This would allow for simple testing before any autonomous behaviour was added as well as acting as a general safety net should anything go catastrophically wrong.

\subsection{Sensing}

By using information received from the attached RGB-D camera, the system had to be be able to interpret the world around it. Specifically, we looked to build up a map of the immediate surrounding area such that it was possible to detect any objects and obstructions.

A method of inferring the position of the robot relative to the rest of the world was also required. Wheeled robots have an advantage in that they are able to apply rotary encoder techniques, allowing them to extrapolate their current position relative to a starting position based on wheel rotations. As this is a walker-style robot, this presented a particularly difficult challenge.

\subsection{Navigation}

Finally, the robot had to be capable of some autonomous behaviour. For this, autonomous navigation was chosen as it would be a stepping stone to any further behaviours. By using the interpreted visual data, the robot had to be able to navigate its environment to reach some particular target while avoiding any obstructions and obstacles where appropriate. To achieve this, we would require a complete working integration of the aforementioned goals.

%%%%%%%%%%%%%%%%%%%%%%%%%%%%%%%%%%%%%%%%%%%%%%%%%%%%%%%%%%%%%%%%%%%%%%%%%%%%%%%%%%%%%%%%%%%%%%%%%%%%

\section{Architecture}

The package concept provided by ROS allows us to divide the control system into a number of dedicated subsystems based on the above requirements. This provides us with a ``separation of concerns'' such that each subsystem only needs to handle one particular task, assuming it has appropriate inputs. For example, the sensing subsystem can assume that it will be provided with valid imagery from the camera thus allowing it to generate a map of the environment. These subsystems can then operate in unison, communicating with each as necessary, to implement the required behaviours. The diagram shown in \autoref{fig:packages} shows an overview of the desired package breakdown. 

\begin{figure}[h!]
    \centering
    \includegraphics[width=16cm]{packages.png}
    \caption{A diagram showing an overview of the packages layout used in the robot control system.}
    \label{fig:packages}
\end{figure}

The particulars of each subsystem would be implemented as a number of nodes within the package of that subsystem, again only carrying out one singular task of that subsystem providing us with another layer of concern separation. Breaking down tasks in this manner, besides being a standard software development practise, allows us to use as many existing ROS packages as possible. Generic implementations of particular functionalities can be taken and adapted as necessary, as long as we supply them with the correct inputs.
\chapter{Implementation}

%%%%%%%%%%%%%%%%%%%%%%%%%%%%%%%%%%%%%%%%%%%%%%%%%%%%%%%%%%%%%%%%%%%%%%%%%%%%%%%%%%%%%%%%%%%%%%%%%%%%

This chapter will document the implementation process of the robot's systems.

%%%%%%%%%%%%%%%%%%%%%%%%%%%%%%%%%%%%%%%%%%%%%%%%%%%%%%%%%%%%%%%%%%%%%%%%%%%%%%%%%%%%%%%%%%%%%%%%%%%%

\section{Hardware Operation}

% FIXME: Maybe put this elsewhere?
From the outset, it was known that the depth sensor relied on the \emph{OpenNI2} library and already had a supporting package. However, no such packages existed for the servo controller.

\subsection{Camera Driver}
\subsection{Servo Driver}

%%%%%%%%%%%%%%%%%%%%%%%%%%%%%%%%%%%%%%%%%%%%%%%%%%%%%%%%%%%%%%%%%%%%%%%%%%%%%%%%%%%%%%%%%%%%%%%%%%%%

\section{Locomotion}

\subsection{Limb Controller}
\subsection{Limb Calibration Tool}
\subsubsection{Usage}

\subsection{Tripod Gait Walker}
\subsection{Joystick Controller}

%%%%%%%%%%%%%%%%%%%%%%%%%%%%%%%%%%%%%%%%%%%%%%%%%%%%%%%%%%%%%%%%%%%%%%%%%%%%%%%%%%%%%%%%%%%%%%%%%%%%

\section{Sensing}

openni2 originally, ccny\_rgbd \cite{ccny_rgbd} provides some clean up features.

\subsection{Visual Odometry}

ccny\_rgbd \cite{ccny_rgbd} was used.

\subsection{Environment Mapping}

ocotomap

\subsubsection{Alternatives}

SLAM, but expects laser scan.

%%%%%%%%%%%%%%%%%%%%%%%%%%%%%%%%%%%%%%%%%%%%%%%%%%%%%%%%%%%%%%%%%%%%%%%%%%%%%%%%%%%%%%%%%%%%%%%%%%%%

\section{Navigation}

Built in stack.

\subsection{Path Planning}

\chapter{Evaluation}
\label{chap:evaluation}

%%%%%%%%%%%%%%%%%%%%%%%%%%%%%%%%%%%%%%%%%%%%%%%%%%%%%%%%%%%%%%%%%%%%%%%%%%%%%%%%%%%%%%%%%%%%%%%%%%%%

\section{Experimental Design}

A number of experiments were ran to evaluate the correctness of this robot control system.

\subsection{Hardware Interaction}

These experiments were created to evaluate that the systems used to control the hardware used in the robot operated correctly.

Check one servo.

Check multiple servos.

Check ranges.

Check timings.

Video feeds from Xtion.

\subsection{Locomotion}

These experiments were created to evaluate that the systems designed to handle locomotion operated correctly.

Calibration tool.

Calibrated offsets applied.

Walk cycle (linear).

Walk cycle (rotation).

Walk cycle (combination). 

\subsection{Sensing}

These experiments were created to evaluate that the systems could interpret the world around the robot via the supplied sensory input.

Visual odometry (manual movement of camera).

Visual odometry (walk cycle).

Mapping (stationary).

Mapping (object in front).

Mapping (while walking).

\subsection{Navigation}

These experiments were created to evaluate that the systems could navigate autonomously using the interpreted sensory information.

Navigation accuracy. (no obstacles, straight line)

Navigation accuracy. (obstacle in way, curved path)

Obstacles in way during navigation.

\section{Results}

\subsection{Hardware Interaction}

Issues with timing, seems to be host dependent.

\subsection{Locomotion}

Difficulty in calibrating by eye alone. 

Servos are inaccurate with small changes.

Maximum speed constraints.

\subsection{Sensing}

Visual odometry drift. Difficulty with stationary rotation. Interesting reaction to mirrors. No resync if location is lost.

Mapping is awesome all of the time, ruined by visual odometry.

\subsection{Navigation}

It works alright, I guess.
\chapter{Conclusion}
\label{chap:conclusion}

This chapter concludes the report, reviewing our initial objectives and providing an assessment of over all success against these objectives. The chapter will then end with a discussion about any future developments that could be made with the robot and robot control system.

%%%%%%%%%%%%%%%%%%%%%%%%%%%%%%%%%%%%%%%%%%%%%%%%%%%%%%%%%%%%%%%%%%%%%%%%%%%%%%%%%%%%%%%%%%%%%%%%%%%%

\section{Summary}

The original premise of this project was develop a control system for a hexapod-type robot through the use of the Robot Operating System (ROS). A number of complex behaviours such as environmental mapping and autonomous navigation would be implemented. By doing this, we would evaluate the usefulness of ROS as a means for rapidly developing these control systems in a simplified manner. The large variety of standard and community-provided ROS packages would be explored, and we would attempt to leverage these packages to endow the robot with the complex behaviours described. These packages should have allowed us to implement these behaviours without requiring a deep understanding of their inner workings.

This control system was implemented by dividing the system into a number of subsystems. These subsystems would implement hardware operations, locomotion facilities, sensing facilities and autonomous navigation facilities. While a number of standard and community-provided packages existed which we utilised to implement these subsystems, specifically for the sensing and navigation subsystems, we still had to implement a number of platform specific nodes. These nodes would provide drivers for the servo controller, a means of calibrating the servos, and a means of locomoting the robot through a tripod walk gait. ROS proved to be very useful in simplifying the development of these custom nodes.

An evaluation of the control system showed that the implemented system was fairly functional. All of the nodes which were developed specifically for the our hexapod-type robot worked as intended.  The visual odometery system was shown to drift quite intensely as the robot performed angular motions but was otherwise reasonably accurate in estimating the robot's position in space. The mapping system gave a good facsimile of the surrounding environment, but drift from the visual odometry system tended to cause severe distortions in the resulting generated map. The autonomous navigation system was particularly good at moving the robot along direct paths, even when an obstacle was placed in its path. However, the system was not particularly good at moving along complex paths---i.e., around large obstacles---especially in an unknown environment. The control system tended to take a long time to get the robot to the correct position, and in some cases failed completely.

Overall, the project has been relatively successful. It would have been very difficult to complete this project without the use of ROS. The many standard and community-provided packages allowed us to leverage complex functionalities that would have otherwise required many person hours to implement by hand. Each subsystem could have been considered an entire project of its own right without these packages. Additionally, the development of the custom nodes which were required was accelerated through the use of ROS in general.

%%%%%%%%%%%%%%%%%%%%%%%%%%%%%%%%%%%%%%%%%%%%%%%%%%%%%%%%%%%%%%%%%%%%%%%%%%%%%%%%%%%%%%%%%%%%%%%%%%%%

\section{Further Work}

This final section will detail a number of ways in which the project could be developed further. Much of this additional functionality could be provided by existing standard or community-provided ROS packages, however further research is necessary to prove the adequacy of such packages. Furthermore, some of these suggestions require additional or even entirely new hardware. These would require new drivers, control software, not to mention the hardware integration itself.

\subsection{Improved Actuators}
The actuators currently used in the robot severely limit accuracy and movement speed. While these servos are extremely cheap and have incredible amounts of torque, this comes at the cost of accuracy, speed and build quality. Generally, the robot requires complete calibration every time it is powered on to ensure correct operation, and even after the calibration process it isn't particularly accurate.

Replacing these servos with ones of higher quality would allow the robot to move much more quickly and accurately. As a result, the walking motions implemented by the tripod gait would be smoother in general. This could possibly help with the drift issues occurring in the visual odometry system.

\subsection{Inverse Kinematics}
To improve movements in general, an inverse kinematics system could be added. This system would allow us to specify where the legs should be positioned relative to the ground, rather than implementing a specific sequence for the angles to rotate to. The inverse kinematics system could then calculate which angle to rotate the joints to based on these position. This would allow for much more smoother movements in terms of the walk gait, which may be beneficial to the visual odometry system as mentioned previously.

This allows for much more complex movements over all. For example, the base of the hexapod can be pivoted but in such a way that the legs remain in place. With an inverse kinematics system, we could supply commands that specify this base rotation having to be concerned about the underlying actuator movements. This can then be expanded to implement more complex walking gaits, perhaps even to allow the robot to walk up stairs, for example.

\subsection{Additional Sensor Hardware \& Environment Interpretation}
Additional sensory input would allow the control system to understand the environment surrounding the robot much more clearly. In particular, a pair of accelerometer and gyroscope sensors would be extremely helpful. These sensors would provide data relating to the acceleration and orientation of the robot in 3D space. By combining data from these sensors with the output given by the visual odometry system, it may be possible to improve the accuracy of the positional fix. Furthermore, it may be possible to automatically calibrate the servos by using these sensors as it can be used to give a reading as to when the robot is in a stable upright position.

The RGB-D camera used in the robot---specifically the \emph{ASUS Xtion Pro Live}---also features a pair of microphones. These could be used in some manner to implement voice recognition commands, for example. In particular, it is possible to infer from which direction any sound is coming by comparing times between the input signals of the two microphones. This could be used to implement some sort of follow behaviour based on sound.

\subsection{Untethered \& Cloud Operation}
The robot is only capable of tethered operation in its current hardware configuration. Specifically, a large bundle of cables protrudes from the back of the robot, connecting the RGB-D camera and servo controller to a nearby computer and power source. This restricts the robot to 5 meter radius around the equipment, as USB devices tend to stop functioning correctly with cables above this length without active boosters. 

The decision to limit to tethered operation only was to, primarily, reduce the complexity of the robot hardware itself while the control system was being developed. Batteries require additional circuitry, such as voltage converters and charging circuits. Furthermore, the power requirements for the robot are quite high posing somewhat of a potential health risk, as the robot draws 10A at 6V at full operational speed. Now that a working control system is implemented, these issues are no longer a concern.

Furthermore, the sensing system requires a particularly high performance machine to achieve good results. Even if it were possible to interface with the RGB-D sensor through a microcontroller, it would in no shape or form be able to meet these performance demands.

However, it would be possible to exploit the distributed nature of ROS to alleviate these issues. A Beaglebone Black, for example, could be attached to the robot. This could connect directly to the RGB-D camera and servo controller over USB as the current system does. Subsystems that interface with the hardware would run on this embedded system. Another machine could then be dedicated to performing the complex processing tasks.

This could be further expanded to make use of cloud services such as Amazon EC2 and Microsoft Azure. Rather than running the performance demanding nodes on a physical machine, they could be ran on a number of virtual machines. 
\begin{appendices}

\chapter{Demonstration Video}

A demonstration video of the robot can be found at \url{http://youtu.be/efLMad5wh7w}.

\chapter{Source Code}

A repository containing the source code of the implemented robot control system can be found on GitHub at \url{https://github.com/Knifa/HexapodKit}.

\chapter{Evaluation Results}

This appendices show the full results from the control system evaluation.

\section{Visual Odometery}

These tables contain the resulting $(x, y)$ position and $\theta_z$ angle reported by the visual odometery system after various movements. The positions are relative to the robot's starting position---i.e., the starting position is $(0, 0)$.

\begin{table}[!h]
	\centering
	\begin{tabular}{ r r r }
		\toprule
		\multicolumn{3}{c}{\textbf{Manual}} \\
		\midrule
		\textbf{$x$} & \textbf{$y$} & \textbf{$\theta_z$} \\
		\midrule
		$0.905$m &
		$0.048$m &
		$3.565$\textdegree{} \\

		$0.906$m &
		$0.007$m &
		$-0.709$\textdegree{} \\

		$0.932$m &
		$0.024$m &
		$3.648$\textdegree{} \\

		$0.932$m &
		$-0.007$m &
		$2.786$\textdegree{} \\

		$0.928$m &
		$-0.006$m &
		$-3.599$\textdegree{} \\

		$0.938$m &
		$-0.010$m &
		$-2.813$\textdegree{} \\

		$0.928$m &
		$0.028$m &
		$2.752$\textdegree{} \\

		$0.932$m &
		$-0.020$m &
		$-3.681$\textdegree{} \\

		$0.934$m &
		$-0.028$m &
		$0.782$\textdegree{} \\

		$0.909$m &
		$0.005$m &
		$1.981$\textdegree{} \\

		\midrule
		$0.924$m &
		$0.004$m &
		$0.471$\textdegree{} \\
		\bottomrule
	\end{tabular}
	\hspace{2ex}
	\begin{tabular}{ r r r }
		\toprule
		\multicolumn{3}{c}{\textbf{Walking}} \\
		\midrule
		\textbf{$x$} & \textbf{$y$} & \textbf{$\theta_z$} \\
		\midrule
		$0.936$m &
		$-0.019$m &
		$-0.871$\textdegree{} \\

		$0.949$m &
		$0.046$m &
		$-6.362$\textdegree{} \\

		$0.944$m &
		$0.039$m &
		$1.971$\textdegree{} \\

		$0.955$m &
		$-0.046$m &
		$5.363$\textdegree{} \\

		$0.968$m &
		$0.057$m &
		$-3.224$\textdegree{} \\

		$0.934$m &
		$0.021$m &
		$-0.548$\textdegree{} \\

		$0.949$m &
		$0.037$m &
		$-0.27$\textdegree{} \\

		$0.987$m &
		$-0.044$m &
		$-2.067$\textdegree{} \\

		$0.968$m &
		$0.039$m &
		$1.594$\textdegree{} \\

		$0.934$m &
		$0.001$m &
		$6.325$\textdegree{} \\

		\midrule
		$0.952$m &
		$0.013$m &
		$0.191$\textdegree{} \\
		\bottomrule
	\end{tabular}

	\caption{Results from the visual odometry system after having moved the robot by $1$m both by hand and using the walk cycle.}
	\label{tab:eval_vo}
\end{table}

\begin{table}[!h]
	\centering
	\begin{tabular}{ r r r }
		\toprule
		\multicolumn{3}{c}{\textbf{Manual}} \\
		\midrule
		\textbf{$x$} & \textbf{$y$} & \textbf{$\theta_z$} \\
		\midrule
		$0.066$m &
		$-0.027$m &
		$90.023$\textdegree{} \\

		$0.088$m &
		$-0.059$m &
		$88.592$\textdegree{} \\

		$0.199$m &
		$0.029$m &
		$90.419$\textdegree{} \\

		$0.097$m &
		$0.069$m &
		$88.109$\textdegree{} \\

		$0.055$m &
		$-0.039$m &
		$88.51$\textdegree{} \\

		$0.176$m &
		$0.074$m &
		$88.438$\textdegree{} \\

		$0.055$m &
		$-0.005$m &
		$91.587$\textdegree{} \\

		$0.14$m &
		$0.056$m &
		$89.128$\textdegree{} \\

		$0.055$m &
		$0.004$m &
		$88.484$\textdegree{} \\

		$0.064$m &
		$-0.058$m &
		$89.676$\textdegree{} \\

		\midrule
		$0.1$m &
		$0.004$m &
		$89.296$\textdegree{} \\
		\bottomrule
	\end{tabular}
	\hspace{2ex}
	\begin{tabular}{ r r r }
		\toprule
		\multicolumn{3}{c}{\textbf{Walking}} \\
		\midrule
		\textbf{$x$} & \textbf{$y$} & \textbf{$\theta_z$} \\
		\midrule
		$0.152$m &
		$0.068$m &
		$88.937$\textdegree{} \\

		$0.283$m &
		$-0.011$m &
		$89.188$\textdegree{} \\

		$0.197$m &
		$-0.046$m &
		$92.523$\textdegree{} \\

		$0.14$m &
		$-0.044$m &
		$85.965$\textdegree{} \\

		$0.119$m &
		$0.01$m &
		$86.522$\textdegree{} \\

		$0.098$m &
		$-0.072$m &
		$85.157$\textdegree{} \\

		$0.239$m &
		$0.021$m &
		$93.074$\textdegree{} \\

		$0.122$m &
		$0.015$m &
		$92.791$\textdegree{} \\

		$0.169$m &
		$-0.044$m &
		$90.951$\textdegree{} \\

		$0.199$m &
		$-0.077$m &
		$87.728$\textdegree{} \\

		\midrule
		$0.172$m &
		$-0.018$m &
		$89.284$\textdegree{} \\
		\bottomrule
	\end{tabular}

	\caption{Results from the visual odometry system after having rotated the robot by $90$\textdegree{} both by hand and using the walk cycle.}
	\label{tab:eval_vo_rot}
\end{table}

\section{Navigation}


\begin{table}[!h]
	\centering
	\begin{tabular}{ c }
		\toprule
		\textbf{Manual} \\
		\midrule
		Time \\
		\midrule
		$34.203$s \\

		$34.411$s \\

		$37.379$s \\

		$34.983$s \\

		$38.032$s \\

		$34.401$s \\

		$34.988$s \\

		$38.167$s \\

		$36.907$s \\

		$38.166$s \\

		\midrule
		$36.164$s \\
		\bottomrule
	\end{tabular}
	\hspace{2ex}
	\begin{tabular}{ c }
		\toprule
		\textbf{Autonomous} \\
		\midrule
		Time \\
		\midrule
		$51.267$s \\

		$45.533$s \\

		$40.938$s \\

		$46.094$s \\

		$42.858$s \\

		$48.690$s \\

		$44.141$s \\

		$45.372$s \\

		$44.847$s \\

		$45.208$s \\

		\midrule
		$45.495$s \\
		\bottomrule
	\end{tabular}
	\caption{Time taken for the robot to autonomously navigate a direct $3$m path.}
	\label{tab:eval_nav_dp_wo}
\end{table}

\begin{table}[!h]
	\centering
	\begin{tabular}{ c }
		\toprule
		\textbf{Manual} \\
		\midrule
		Time \\
		\midrule
		$0$m $50.406$s \\

		$0$m $50.183$s \\

		$0$m $50.782$s \\

		$0$m $51.81$s \\

		$0$m $51.012$s \\

		$0$m $50.406$s \\

		$0$m $52.334$s \\

		$0$m $52.264$s \\

		$0$m $51.475$s \\

		$0$m $50.684$s \\

		\midrule
		$0$m $51.136$s \\
		\bottomrule
	\end{tabular}
	\hspace{2ex}
	\begin{tabular}{ c }
		\toprule
		\textbf{Autonomous} \\
		\midrule
		Time \\
		\midrule
		$1$m $4.771$s \\

		$1$m $8.007$s \\

		$1$m $4.724$s \\

		$1$m $0.816$s \\

		$1$m $7.04$s \\

		$1$m $6.758$s \\

		$1$m $1.963$s \\

		$1$m $8.458$s \\

		$1$m $5.232$s \\

		$1$m $9.0$s \\

		\midrule
		$1$m $5.677$s \\
		\bottomrule
	\end{tabular}
	\caption{Time taken for the robot to autonomously navigate a direct $3$m path after an obstacle has been placed $1$m from the starting position.}
	\label{tab:eval_nav_dp}
\end{table}

\begin{table}[!h]
	\centering
	\begin{tabular}{ c }
		\toprule
		\textbf{Manual} \\
		\midrule
		Time \\
		\midrule
		$2$m $5.643$s \\

		$2$m $10.224$s \\

		$2$m $7.821$s \\

		$2$m $8.206$s \\

		$2$m $4.561$s \\

		$2$m $1.542$s \\

		$2$m $2.964$s \\

		$2$m $2.178$s \\

		$2$m $9.315$s \\

		$2$m $4.037$s \\

		\midrule
		$2$m $5.649$s \\
		\bottomrule
	\end{tabular}
	\hspace{2ex}
	\begin{tabular}{ c }
		\toprule
		\textbf{Autonomous} \\
		\midrule
		Time \\
		\midrule
		$2$m $29.284$s \\

		DNF \\
				
		$2$m $28.418$s \\

		$2$m $34.716$s \\

		$2$m $34.484$s \\

		$2$m $25.634$s \\

		$2$m $26.558$s \\

		$2$m $32.505$s \\

		$2$m $29.575$s \\

		$2$m $33.873$s \\

		\midrule
		$2$m $30.561$s \\
		\bottomrule
	\end{tabular}
	\caption{Time taken for the robot to navigate a known environment, specifically the path shown in \autoref{fig:eval_plan}. Average time taken in this case did not include \emph{DNF} results.}
	\label{tab:eval_nav_cp_known}
\end{table}

\begin{table}[!h]
	\centering
	\begin{tabular}{ c }
		\toprule
		\textbf{Autonomous} \\
		\midrule
		Time \\
		\midrule
		$3$m $47.259$s \\

		$3$m $46.573$s \\

		$3$m $24.283$s \\

		DNF \\

		DNF \\

		$3$m $23.027$s \\

		$3$m $41.999$s \\

		$3$m $32.822$s \\

		$3$m $46.440$s \\

		DNF \\

		\midrule
		$3$m $37.486$s \\
		\bottomrule
	\end{tabular}
	\caption{Time taken for the robot to navigate an unknown environment, specifically the path shown in \autoref{fig:eval_plan}. Average time taken in this case did not include \emph{DNF} results.}
	\label{tab:eval_nav_cp_unknown}
\end{table}

\end{appendices}

%%%%%%%%%%%%%%%%%%%%%%%%%%%%%%%%%%%%%%%%%%%%%%%%%%%%%%%%%%%%%%%%%%%%%%%%%%%%%%%%%%%%%%%%%%%%%%%%%%%%

\renewcommand{\bibname}{References}
\bibliographystyle{plain}
\bibliography{report}

\end{document}
